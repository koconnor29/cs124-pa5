\documentclass[final]{article}
\usepackage{latexsym, amscd, amsfonts, eucal, mathrsfs, amsmath, amssymb, amsthm, xypic,xr, makeidx, stmaryrd,fullpage, enumitem}
\usepackage[all,cmtip]{xy}
 \usepackage{pictexwd,dcpic, dsfont}
\usepackage{graphicx}
%\usepackage[T1]{fontenc}
%\usepackage[sc]{mathpazo}

\linespread{1.05}         % Palatino needs more leading (space between lines)
\makeindex
\newtheorem{theorem}{Theorem}[section]
\newtheorem{lemma}[theorem]{Lemma}
\newtheorem{proposition}[theorem]{Proposition}
\newtheorem{corollary}[theorem]{Corollary}
\newtheorem{de}[theorem]{Definition}
\newtheorem{ex}[theorem]{Example}
\graphicspath{{c:/Users/Gurinder_Dhillon/My_Pictures/}} 
\graphicspath{{c:/Users/Gurinder_Dhillon/Downloads/}} 
\graphicspath{{./images/}}


\newcommand{\im}{\text{im} \;}

\newcommand{\coker}{\text{coker }}



\newenvironment{definition}[1][Definition]{\begin{trivlist}
\item[\hskip \labelsep {\bfseries #1}]}{\end{trivlist}}
\newenvironment{exercise}[1][Exercise]{\begin{trivlist}
\item[\hskip \labelsep {\bfseries #1}]}{\end{trivlist}}
\newenvironment{example}[1][Example]{\begin{trivlist}
\item[\hskip \labelsep {\bfseries #1}]}{\end{trivlist}}
\newenvironment{remark}[1][Remark]{\begin{trivlist}
\item[\hskip \labelsep {\bfseries #1}]}{\end{trivlist}}

\begin{document}
\title{pa3}
\author{Kevin O'Connor, Gurbir Dhillon}
\maketitle

Throughout we maintain the notations introduced in the assignment. Namely, we consider a set of $n$ nonegative integers $a_i$, and we set $b := \sum_i a_i$.

\section{Dynamic Programming}

We first present a dynamic programming solution to the number partition problem, in $O(nb)$ time. Consider first the related `subset problem' $S(k)$ of whether there exists a subset $I \subseteq \{1, \ldots, n\}$ such that $$\sum_{i \in I} a_i = k$$that is, there exists a subset which sums to $k$, for $0 \leqslant k \leqslant b$. 


\begin{lemma}
$S(k)$ can be solved in $O(nb)$ time, simultaneously for all $k$, $0 \leqslant k \leqslant b$. 

\begin{proof}
Write $S(i,k)$ for the problem restricted to the first $i$ numbers: does there exists a subset $J \subseteq \{1, \ldots, i\}$ such that $$\sum_{j \in J} a_j = k$$with output $T$ or $F$.

We have the initial condition: $$S(1,k) = \begin{cases} T & a_1 = k \\ F & a_1 \neq k \end{cases}$$and the recursion$$S(i,k) = \begin{cases} T & S(i-1,k)  = T \lor (a_i \leqslant k \land S(i-1, k-a_i) = T) \\ F & o.w. \end{cases}$$To see the recursion, given a solution condition on whether $a_i$ (the last number) was used or not: if it was not, then $S(i-1,k)$ was solvable, and if was, we have $a_i \leqslant k$, (recall $a_j \geqslant 0$), and hence a subset of the first $i-1$ numbers sum to $k - a_i$. It's easy to see both arguments are in fact if and only if. 

We have that $S(n,k) = S(k)$, the desired solutions; in running time, initialization and each step of the recursion takes constant time, and we perform $O(nb)$ calculations, one for each $S(i,k), 1 \leqslant i \leqslant n, 0 \leqslant k \leqslant b$, as desired. 



\end{proof}


\end{lemma}

With the lemma, to solve the number partition problem, produce all the $S(k)$ as in the lemma. I claim that minimizing the residue, is equivalent to finding the $k$ closest to $\frac{b}{2}$ for which $S(k)$ is true ($k$ not necessarily unique). 

To see this:

\begin{lemma} There exists a partition with residue $i$ $\Leftrightarrow$ $S(\frac{b+i}{2}) = T$. 

\begin{proof}
For a partition $A_1, A_2$ with residue $i$, write $$s := \sum_{i \in A_1} a_i \quad \quad l := \sum_{i \in A_2} a_i$$WLOG set $s \leqslant l$; informally these are the heights of the smaller and bigger piles, respectively. Note $l+s = b$

$$l - s = i \Leftrightarrow l - (b-l) = i \Leftrightarrow 2l - b = i \Leftrightarrow l = \frac{b+i}{2}$$as desired. 
\end{proof}

\end{lemma}

With the lemma, having computed the $S(k)$, let $k_0$ denote the least $k \in \{\frac{b}{2}, \ldots, b\}$ such that 
$S(k_0)$ is true, the corresponding minimal residue is $2k_0 - b$, by the lemma. 

\section{Karmarkar-Karp}
We first briefly describe how to implement Karmarkar-Karp on $\{a_1, \ldots, a_n\}$ in $O(n \log n)$ time. 

Create a binary heap $Leftovers$ and an array $Untouched$, initialize $Leftovers$ to empty and $Untouched$ to $\{a_1, \ldots, a_n\}$. Sort the $a_i$ in $O(n \log n)$ time.


At the $i^{th}$ step, find the maximal 2 elements of $Leftovers \coprod Untouched$, it takes constant time to extract the 2 greatest elements of $Leftovers$ and $Untouched$ each, as $Leftovers$ is a binary heap.

Remove $m_1, m_2$ from $Leftovers$ and/or $Untouched$, and then add $|m_1 - m_2|$ to $Leftovers$. 

We perform $n-1$ steps, observe recursively that the size of $Leftovers$ is bounded above by $n$, and we perform maximally 2 removals and 1 insert from it in each step, each step takes $O(\log n)$ time;  at the end, we return the unique element remaining of $Leftovers$, hence this takes $$O(n\log n) + nO(\log n) = O(n \log n)$$ time, as desired. 



\end{document}

01100001011011100010000001000001 


010001110110100101110110011001010010000001101101011001010010000001100001011011100010000001000001


